\documentclass{report}

\usepackage[utf8]{inputenc}
\usepackage[francais]{babel}
\renewcommand{\thesection}{\arabic{section}}
\begin{document}
\title{%
    \begin{minipage}\linewidth
        \centering
        Rapport de Projet d'Algorithmique
        \vskip 3pt
        \large Création d'un Clone d'Hexxagon
    \end{minipage}
}
\author{Guillaume \bsc{Ryckart} - Pablo \bsc{Bourdelas}}
\maketitle

\section{Introduction}

Le but de ce projet est de créer un clone d'hexxagon en C. Le principe du jeu est le suivant. Le tableau est composé de cases hexagonles, sur lequelles deux joueurs débutent avec 2 pions chacun.\\

A chaque tour, un joueur peut soit: copier un de ses pions sur une case vide adjacente, soit déplacer un de ses pions sur une case située à 2 de distance. Dans les deux cas, les pions ennemis adjacents sont pris.\\

De plus, le programme devra également: \begin{itemize}
\item Avoir une interface graphique
\item Gérer des plateaux de différentes tailles
\item Proposer un mode deux joueurs et un mode contre l'odinateur
\item Permettre de bloquer des cases sur le plateau 
\item Mémoriser les scores les plus élevés
\end {itemize}
\newpage
	
\section{Structure du programme}

	\subsection{Gestion de plateau}
		Le plateau est stocké sous forme d'un tableau.
		Chaque case du plateau est représenté par une structure hexa, qui contient une variable énumérée (qui correspond au contenu de la case) ainsi qu'un tableau de pointeurs sur des cases, qui correspondent aux cases voisines directes de celle-ci.\\
		Le type énuméré prend donc 4 valeurs différentes: 	\begin{itemize}
									\item Joueur 1
								  	\item Joueur 2
								  	\item Vide
								  	\item Bloqué
								  	\end{itemize}
	
		Cette structure permet donc de gérer le bloquage de certaines cases, ainsi qu'un calcul facile des cases ou le joeur peut jouer, ou non. 

	\subsection {Boucle principale}
		
	La boucle caputre les clics de manière non bloquante: elle récupère les déplacements de la souris et valide la position au moment du clic. A partir de celle-ci, sur un clic gauche, elle détermine la position de l'hexagone dans la grille. Suite au second clic gauche, elle valide la seconde position, et determine le mouvement a partir de ces deux positions. Un clic dorit annule la sélection du pion courant.\\

Après la determintation et l'application du coup entré, le jeu vérifie si la partie est finie, via la fonction Update\_Alive, puis passe la main a l'autre joueur. 
	\subsection{Vivacité des pions}

Afin de vérifier si la partie est finie, on utilise un principe de vivacité des pions: un pion mort est un pion qui pourra plus se déplacer de toute la partie. On tient une liste des pions vivants, qui est mise a jour a la fin de chaque tour. Il suffit donc de regarder si il reste un déplacement possible pour un pion. Il devient donc facile de vérifier si le jeu est fini: Quand un joueur n'a plus de pions vivants, la partie est finie.\\

Cette méthode a l'avantage d'être moins couteuse en temps que de vérifier case par case, en effet, on ignore les cases vides, et il suffit de faire des comparaisons simples sur le plateau pour determiner si le pion est vivant ou non.

	\subsection{Déplacement et coups}
	Pour determiner où le joueur peut se copier/sauter, on utilise un algoritmhe qui nous renvoie, suivant la case de départ et celle d'arrivée, le coup joué.\\
	
	Pour cela, l'algorithme va calculer un vecteur de déplacement. Ce vecteur rassemble la distance entre les deux cases, ainsi que comment elles sont orientées l'une par rapport a l'autre.\\
	
	Au vu de vecteur il sera donc simple de déterminer si la case est joauble, et, ci c'est le cas, le type de coup jouée dans cette case.\\
	Si c'est une copie, il suffit de changer la valeur de la case, sinon il suffit de déplacer le pion de la case de départ dans celle d'arrivée. Dans les deux cas, on regardera les pions adjacents a la case d'arrivée, et on les convertira dans la couleur du joueur qui vient de jouer
	\subsection{Gestion des scores}
	
	A l'issue d'une partie, le nombre de pions possédés par le vainqueur est sauvegardé, ainsi que le pseudo rentré au début du jeu, dans un ficher .crxx.
	Cela permet d'enregistrer et d'afficher les scores les plus hauts atteints par les utilisateurs. Si l'IA remporte la partie, elle ne sauvegarde pas de score. 

	\subsection{Intelligence Artificielle}
	
	L'intelligence artificelle fonctionne sur le principe du "plus grand score immédiat": A chaque tour l'IA va rechercher pour tous les coups, celui qui lui donne le plus de points, et jouer ce coup.
	
\newpage
\section{Graphismes}
	\subsection{Méthode choisie}
	
		Nous avons choisi de modéliser tout le jeu en utilisant la bibliothèque graphique SDL 1.2, qui nous permet de gérer le son, les graphismes, ainsi que la souris. On définit plusieurs découpes du fichier contenant les sprites via la strucutre SDL\_Rect (cette structure permet de définir un rectangle simple). On peut donc afficher seulment certaines parties du fichier png, et stocker toutes les images sur un seul fichier.\\
		
		Ces découpes sont parfois stockées sous formes de tableaux, comme par exemple:\begin{itemize}
		\item HexaT, qui contient l'Hexagone de base, ainsi que son contour vert ou rouge,
		\item Token, qui contient les Pions des deux joueurs,
		\item winText, qui contient les textes de victoire des deux joueurs, et le texte en cas d'égalité.
		\end {itemize}
		
	\subsection{Style graphique}
	
		Nous avons choisi de faire notre hexxagon dans le style synthwave (Tron Leagacy, Kung Fury, Far Cry : Blood Dragon): Couleurs néon sur du noir, et personnages cyberpunk.

\end{document}

